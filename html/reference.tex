\documentclass[a4paper,11pt,twoside]{book} 
\usepackage{hyperref,fancyhdr,graphicx,amssymb,amsmath,times,makeidx,listings} 
\graphicspath{{img/}} 
\pagestyle{fancyplain} 
\lhead[\fancyplain{}{\textbf\thepage}]{\fancyplain{}{\rightmark}} 
\rhead[\fancyplain{}{\leftmark}]{\fancyplain{}{\textbf\thepage}} 
\cfoot{} 
\setlength{\textwidth}{5.875in} 
\setlength{\parindent}{0pc} 
\setlength{\hoffset}{-0.8cm} 
\setcounter{tocdepth}{1} 
\sloppy{} 
\title{\fbox{\parbox{\textwidth}{\begin{center}\vspace*{2cm}\includegraphics[width=12cm]{logo3.jpg}\\\vspace*{1cm}{\Huge \textbf{The Handbook}\\{\small Version 1.5.8.6}\\\vspace*{1cm}}\end{center}}}} 
\author{\Large \bf David Tschumperl\'e} 
\renewcommand\indexname{Index of commands} 
\makeindex 
\lstset{columns=fullflexible,basicstyle=\normalfont} 
\begin{document} 
\maketitle 
\tableofcontents 
\chapter*{Preamble} 
\section*{License} 
This document is distributed under the \textbf{GNU Free Documentation License}, version 1.3.\\ 
Read the full license terms at \texttt{http://www.gnu.org/licenses/fdl-1.3.txt}.\\~\\ 
An online version of this documentation is available at:\\\texttt{http://gmic.sourceforge.net/reference.shtml}. 
\section*{Motivations} 
G'MIC is an open and full-featured framework for image processing, providing several different user interfaces to 
convert/manipulate/filter/visualize generic image datasets, from 1d scalar signales to 3d+t sequences of multi-spectral volumetric images. 
Technically speaking, what it does is: 
\begin{itemize} 
\item Define a lightweight but powerful script language (the G'MIC language) dedicated to the design of image processing pipelines. 
\item Provide several user interfaces embedding the corresponding interpreter: 
\begin{itemize} 
\item A command-line executable 'gmic', to use the G'MIC framework from a shell. 
In this setting, G'MIC may be seen as a direct (and friendly) competitor of the ImageMagick or GraphicsMagick software suites. 
\item A plug-in 'gmic\_gimp', to bring G'MIC capabilities to the GIMP image retouching software. 
\item A web-service 'G'MIC Online', to allow users applying image processing algorithms directly in a web brower. 
\item A Qt-based interface 'ZArt', for real-time manipulation of webcam images. 
\item A C++ library 'libgmic', to be linked with third-party applications. 
\end{itemize} 
\end{itemize} 
G'MIC is focused on the design of possibly complex pipelines for converting, manipulating, filtering and visualizing generic 1d/2d/3d multi-spectral image datasets. This includes of course color images, but also more complex data as image sequences or 3d(+t) volumetric float-valued datasets.\\ 
 
G'MIC is an open framework: the default language can be extended with custom G'MIC-written commands, defining thus new available image filters or effects. By the way, G'MIC already contains a substantial set of pre-defined image processing algorithms and pipelines (more than 1000).\\ 
 
G'MIC has been designed with portability in mind and runs on different platforms (Windows, Unix, MacOSX). It is distributed under the CeCILL license (GPL-compatible). Since 2008, it is developed in the Image Team of the GREYC laboratory, in Caen/France, by permanent researchers working in the field of image processing on a daily basis. 
\section*{Version} 
 
 gmic: GREYC's Magic for Image Computing. 
 
        Version 1.5.8.6, Copyright (c) 2008-2014, David Tschumperl\'e 
        (http://gmic.sourceforge.net) 
\chapter{Usage} 
\small
\begin{lstlisting}
 gmic [command1 [arg1_1,arg1_2,..]] .. [commandN [argN_1,argN_2,..]] 
 
 'gmic' is an open-source interpreter of the G'MIC language, a script-based programming 
  language dedicated to design image processing pipelines. It can be used to convert, 
  manipulate, filter and visualize datasets made of one or several 1d/2d or 3d multi- 
  spectral images. 
 
 This documentation proposes a complete description of the G'MIC language basics and rules.
\end{lstlisting}
\normalsize
\section{Overall context}
\small
\begin{lstlisting}
  - At any time, G'MIC manages one list of numbered (and optionally named) pixel-based 
     images, entirely stored in computer memory. 
  - The first image of the list has indice '0' and is denoted by '[0]'. The second image of 
     the list is denoted by '[1]', the third by '[2]' and so on. 
  - Negative indices are treated in a cyclic way: '[-1]' refers to the last image of the 
     list, '[-2]' to the penultimate one, etc. Thus, if the list has 4 images, '[1]' and '[-3]' 
     both designate the second image of the list. 
  - A named image may be denoted by '[name]' if 'name' uses characters set [a-zA-Z0-9_] and 
     does not start with a number. Image names can be set or reassigned at any moment during 
     the processing pipeline (see commands '-name' and '-input'). 
  - G'MIC defines a set of various commands and substitution mechanisms to allow the design 
     of complex pipelines managing this list of images, in a very flexible way: 
     You can insert or remove images in the list, rearrange image indices, process images 
     (individually or as a group), merge image data together and output image files. 
  - Such a pipeline can be written itself as a custom G'MIC command storable in a custom 
     commands file, which can be re-used afterwards in another bigger pipeline if necessary.
\end{lstlisting}
\normalsize
\section{Image definition and terminology}
\small
\begin{lstlisting}
  - In G'MIC, an image is modeled as a 1d, 2d, 3d or 4d array of scalar values, uniformly 
     discretized on a rectangular/parallelepipedic domain. 
  - The four dimensions of these arrays are respectively denoted by: 
    . 'width', the number of image columns (size along the 'x'-axis). 
    . 'height', the number of image rows (size along the 'y'-axis). 
    . 'depth', the number of image slices (size along the 'z'-axis). 
        The depth is equal to 1 for usual 2d color or grayscale images. 
    . 'spectrum', the number of image channels (size along the 'c'-axis). 
        The spectrum is respectively equal to 3 and 4 for usual RGB and RGBA color images. 
  - There are no size limitations on each image dimensions. Particularly, the number of image 
     slices or channels can be of arbitrary size within the limits of available memory. 
  - The width, height and depth of an image are considered as 'spatial' dimensions, while the 
     spectrum has a 'multi-spectral' meaning. Thus, a 4d image in G'MIC should be most often 
     regarded as a 3d dataset of multi-spectral voxels. Most of the G'MIC commands will stick 
     with this idea (e.g. command '-blur' will blur images only along the 'xyz' axes). 
  - All pixel values of all images of the list have the same datatype. It can be one among: 
    . 'bool': Stands for 'boolean'. Value range is { 0=false | 1=true }. 
    . 'uchar': Stands for 'unsigned char'. Value range is [0,255] (8bits). 
        This type of pixel coding is commonly used to store 8bits/channels RGB[A] images. 
    . 'char': Value range is [-128,127] (8bits). 
    . 'ushort': Stands for 'unsigned short'. Value range is [0,65535] (16bits). 
        This type of pixel coding is commonly used to store 16bits/channels RGB[A] images. 
    . 'short': Value range is [-32768,32767] (16bits). 
    . 'uint': Stands for 'unsigned int'. Value range is [0,2^32-1] (32bits). 
    . 'int': Value range is [-2^31,2^31-1] (32 bits). 
    . 'float': Value range is [-3.4E38,+3.4E38] (32bits). 
        This type of coding is able to store pixels as 32 bits float-valued numbers. This is 
        the default datatype used by G'MIC image processing operations. 
    . 'double': Value range is [-1.7E308,1.7E308] (64bits). 
        This type of coding is able to store pixels as 64 bits float-valued numbers. 
  - Considering pixel datatypes different than 'float' is generally useless, except to force 
     the input/output of image data to a prescribed binary format. Hence, most G'MIC image 
     image processing commands are available only for the default 'float' pixel datatype 
     (see command '-type' if you need to switch to another pixel datatype).
\end{lstlisting}
\normalsize
\section{Items of a processing pipeline}
\small
\begin{lstlisting}
  - In G'MIC, an image processing pipeline is described as a sequence of items separated by 
     the space character ' '. Such items are interpreted and executed from the left to the 
     right. For instance, the expression: 
       'input.jpg -blur 3,0 -sharpen 10 -resize 200%,200% -output output.jpg' 
     defines a valid pipeline composed of nine G'MIC items. 
  - A G'MIC item is a string which represents either a command, a set of command arguments, 
     a filename, or a special input string. 
  - Escape characters '\' and double quotes '"' can be used (as usual) to define items 
     containing spaces, or any other character sequences. For instance, the strings 
     'single\ item' and '"single item"' define the same string item, with a space in it.
\end{lstlisting}
\normalsize
\section{Input data items}
\small
\begin{lstlisting}
  - If a specified G'MIC item appears to be an existing filename, the corresponding image 
     data are loaded and inserted at the end of the image list. 
  - Special filenames '-' and '-.ext' stand for the standard input/output streams, optionally 
     forced to be in a specific 'ext' file format (e.g. '-.jpg' or '-.png'). 
  - The following special input strings may be used as G'MIC items to create and insert new 
     images with prescribed values, at the end of the image list: 
    . '[selection]' or '[selection]xN': Insert 1 or N copies of selected existing images. 
       'selection' may contain one or several images (see next section for details). 
    . 'width[%],_height[%],_depth[%],_spectrum[%],_values': Insert a new image with 
       specified size and values (adding '%' to a dimension means 'percentage of the size 
       along the same axis, taken from the last image '[-1]''). Any specified dimension 
       can be also written as '[image]', and is then set to the size (along the same axis) 
       of the existing specified image [image]. 'values' can be either a sequence of numbers 
       separated by commas ',', or a mathematical expression, as e.g. in input item 
       '256,256,1,3,if(c==0,x,if(c==1,y,0))' which creates a 256x256 RGB color image with a 
       spatial shading on the red and green channels. 
    . '(v1,v2,..)': Insert a new image from specified prescribed values. 
       Value separator inside parentheses can be ',' (column separator.), ';' (row sep.), 
       '/' (slice sep.) or '^' (channel sep.). For instance, expression 
       '(1,2,3;4,5,6;7,8,9)' creates a 3x3 matrix (scalar image), with values from 1 to 9. 
    . '0': Insert a new 'empty' image, containing no pixel data. Empty images are used only 
       in rare occasions. 
  - Input item 'name=value' declares a new local or global variable 'name', or assign a new 
     value to an existing variable. Variable names use characters set [a-zA-Z0-9_] and cannot 
     start with a number. A variable definition is always local to the current command except 
     when it starts by the underscore character '_'. In that case, it becomes also accessible 
     by any command invoked outside the current command scope.
\end{lstlisting}
\normalsize
\section{Command items and selections}
\small
\begin{lstlisting}
  - A G'MIC item starting by '-' designates a command, most of the time. Generally, commands 
     perform image processing operations on one or several available images of the list. 
  - Common commands have two equivalent names (regular and short). For instance, command 
     names '-resize' and '-r' refer to the same image resizing action. 
  - A G'MIC command may have mandatory or optional arguments. Command arguments must be 
     specified in the next item on the command line. Commas ',' are used to separate multiple 
     arguments, if any required. 
  - The execution of a G'MIC command may be restricted only to a subset of the image list, by 
     appending '[subset]' to the command name. Examples of valid syntaxes for 'subset' are: 
    . '-com[0,1,3]': Apply command only on images [0],[1] and [3]. 
    . '-com[3-5]': Apply command only on images [3] to [5] (i.e, [3],[4] and [5]). 
    . '-com[50%-100%]': Apply command only on the second half of the image list. 
    . '-com[0,-4--1]': Apply command only on the first and the four latest images. 
    . '-com[0-9:3]': Apply command only on images [0] to [9], with a step of 3 
                      (i.e. on images [0], [3], [6] and [9]). 
    . '-com[0--1:2]': Apply command only on images of the list with even indices. 
    . '-com[0,2-4,50%--1]': Apply command on images [0],[2],[3],[4] and on the second half 
                             of the image list. 
    . '-com[^0,1]': Apply command on all images except the first two. 
    . '-com[name1,name2]': Apply command on named images 'name1' and 'name2'. 
  - Indices in selections are always sorted in increasing order, and duplicate indices are 
     discarded. For instance, selections '[3-1,1-3]' and '[1,1,1,3,2]' are both equivalent to 
     '[1-3]'. If you want to repeat a single command multiple times on an image, use a 
     '-repeat..-done' loop. Inverting the order of images in a selection can be achieved by 
     inverting first the order of the images in the list, with command '-reverse[selection]'. 
  - G'MIC commands invoked without '[subset]' are applied on all images of the list. 
  - A G'MIC command starting with '--' instead of '-' does not act 'in-place' but inserts its 
     result as one or several new images at the end of the image list. 
  - There are two different types of commands that can be run by the G'MIC interpreter: 
    . Native commands, are hard-coded functionalities in the interpreter core. 
       They are thus compiled as machine code and run quickly, most of the time. 
       Omitting an argument when invoking a native command is not permitted, except if all 
       following arguments are also omitted. For instance, call to '-plasma 10,,5' is invalid 
       but '-plasma 10' is correct. 
    . Custom commands, are defined as G'MIC pipelines of native or custom commands. 
       They are interpreted by the G'MIC interpreter, and run slower than native commands. 
       But omitting arguments when invoking a custom command is permitted. For instance, 
       expressions '-flower ,,,100,,2' or '-flower ,' are correct. 
  - A user may easily add its own custom commands to the G'MIC interpreter (see section 
     'Adding custom commands'). Native commands cannot be added unless you modify the G'MIC 
     interpreter source code.
\end{lstlisting}
\normalsize
\section{Inputs/outputs properties}
\small
\begin{lstlisting}
  - G'MIC is able to read/write most of the classical image file formats, including: 
    . 2d grayscale/color files: .png, .jpeg, .gif, .pnm, .tif, .bmp, .. 
    . 3d volumetric files: .dcm, .hdr, .nii, .pan, .inr, .pnk, .. 
    . Image sequences: .mpeg, .avi, .mov, .ogg, .flv, .. 
    . Generic ascii or binary data files: .cimg, .cimgz, .dlm, .asc, .pfm, .raw, .txt, .h. 
    . 3d object files: .off. 
  - When dealing with color images, G'MIC generally reads, writes and displays data using the 
     usual RGB color space. 
  - G'MIC is able to manage 3d objects that may be read from files or generated by G'MIC 
     commands. They are stored as one-column scalar images containing the object data, in the 
     following order: { magic_number; sizes; vertices; primitives; colors; opacities }. 
     These 3d representations can be processed as regular float-valued images. 
     (see command '-split3d' for accessing each of these 3d object data separately). 
  - Be aware that usual file formats may be sometimes not adapted to store all the available 
     image data, since G'MIC uses float-valued coding of image pixels. For instance, saving 
     an image that was initially loaded as a 16bits/channel image, as a .jpg file will result 
     in loss of information. Use the .cimg file extension (or .cimgz, its compressed 
     version) to ensure that all data precision will be preserved when saving images. 
  - File options can/must be set for these specific file formats: 
    . Video files: Only sub-frames of an image sequence may be loaded, using the input 
       expression 'filename.ext,[first_frame[%][,last_frame[%][,step]]]'. 
       Output framerate and bitrate (in Kb/s) can be also set by using the output expression 
       'filename.mpg,_fps,_bitrate'. 
    . .cimg[z] files : Only crops and sub-images of .cimg files can be loaded, using the input 
      expressions 'filename.cimg,N0,N1', 'filename.cimg,N0,N1,x0,x1', 
      'filename.cimg,N0,N1,x0,y0,x1,y1', 'filename.cimg,N0,N1,x0,y0,z0,x1,y1,z1', 
      'filename.cimg,N0,N1,x0,y0,z0,c0,x1,y1,z1,c1'. 
      Specifying '-1' for one coordinates stands for the maximum possible value. 
      Output expression 'filename.cimg[z][,datatype]' can be used to force the output pixel type. 
      'datatype' can be { bool | uchar | char | ushort | short | uint | int | ulong | long | float | double }. 
    . .raw binary files: Image dimensions and input pixel type may be specified when loading 
       .raw files with input expresssion 'filename.raw[,datatype][,width][,height[,depth[,dim]]]]'. 
       If no dimensions are specified, the resulting image is a one-column vector with 
       maximum possible height. Pixel type can also be specified with the output 
       expression 'filename.raw[,datatype]'. 
       'datatype' can be { bool | uchar | char | ushort | short | uint | int | ulong | long | float | double }. 
    . .yuv files: Image dimensions must be specified, and only sub-frames of an image 
       sequence may be loaded, using the input expression 
      'filename.yuv,width,height[,first_frame[,last_frame[,step]]]'. 
    . .tiff files: Only sub-images of multi-pages tiff files can be loaded, using the input 
       expression 'filename.tif,[first_frame,[last_frame,[step]]]'. 
       Output expression 'filename.tiff,[datatype[,compression]]' can be used to specify the 
       output pixel type, as well as the compression method. 'compression' can be 
       { 0=none | 1=CCITTRLE | 2=CCITT4 | 3=CCITT6 | 4=LZW | 5=JPEG1 | 6=JPEG2 }. 
      'datatype' can be { bool | uchar | char | ushort | short | uint | int | ulong | long | float | double }. 
    . .gif files: Animated gif files can be saved, using the input expression 
      'filename.gif,fps,nb_loops'. 
      Specify 'nb_loops=0' to get an infinite number of animation loops. 
    . .jpeg files: The output quality may be specified (in %), using the output expression 
       'filename.jpg,30' (here, to get a 30% quality output). 
    . .mnc files: The output header can set from another file, using the output expression 
       'filename.mnc,header_template.mnc'. 
    . .pan, .cpp, .hpp, .c and .h files: The output datatype can be selected with output expression 
       'filename[,datatype]'. 
      'datatype' can be { bool | uchar | char | ushort | short | uint | int | ulong | long | float | double }. 
    . .gmic files: These filenames are assumed to be G'MIC custom commands files. Loading 
       such a file will add the commands it defines to the interpreter. Debug infos can be 
       enabled/disabled by the input expression 'filename.gmic,add_debug_infos={ 0 | 1 }'. 
    . Inserting 'ext:' on the beginning of a filename (e.g. 'jpg:filename') forces G'MIC to 
       read/write the file as it would have been done if it had the specified extension. 
  - Some input/output formats and options may not be supported by your current version of 
     'gmic', depending on the configuration flags set for the build of the 'gmic' binaries.
\end{lstlisting}
\normalsize
\section{Substitution rules}
\small
\begin{lstlisting}
  - G'MIC items containing '@', '$' or '{}' may be substituted before being interpreted. Use 
     the substituting expressions below to access data from the interpreter environment: 
    . '@#' is substituted by the current number of images in the list. 
    . '@*' is substituted by the number of available cpus. 
    . '@.' is substituted by the current version number of the G'MIC interpreter 
    . '@^' is substituted by the current verbosity level. 
    . '@%' is substituted by the pid of the current process. 
    . '@|' is substituted by the current value (expressed in seconds) of a millisecond 
       precision timer. 
    . '@?' is substituted by the current data type of image pixels. 
    . '@/' is substituted by the current number of levels in the command scope. 
    . '@{/}' or '@{/,subset}' are substituted by the content of the global scope, or a 
       subset of it. If specified subset refers to multiple scope items, they are separated 
       by slashes '/'. 
    . '@>' and '@<' are equivalent. They are both substituted by the number of nested 
       'repeat-done' loops that are currently running. 
    . '@{>}' or '@{>,subset}' are substituted by the indice values (or a subset of them) of 
       the running 'repeat-done' loops, expressed in the ascending order, starting from 0. 
       If specified subset refers to multiple indices, they are separated by commas ','. 
    . '@{<}' or '@{<,subset}' do the same but in descending order. 
    . '@indice' or '@{indice,feature}' are substituted by the list of pixel values of the 
       image [indice] (separated by commas), or by a specific feature (or subset) of it. 
       'indice' can be an indice or an image name. Requested 'featured' can be one of: 
         . 'w': image width (number of image columns). 
         . 'h': image height (number of image rows). 
         . 'd': image depth (number of image slices). 
         . 's': image spectrum (number of image channels). 
         . 'wh': image width x image height. 
         . 'whd': image width x image height x image depth. 
         . 'whds': image width x image height x image depth x image spectrum. 
                   (i.e. number of values in the specified image, eq. to '#'). 
         . 'r': image shared state (1, if the pixel buffer is shared, 0 otherwise). 
         . 'n': image name or filename (if the image has been read from a file). 
         . 'b': image basename (i.e. filename without the folder path nor extension). 
         . 'x': image extension (i.e last characters after the last '.' in the filename). 
         . 'f': image folder name. 
         . '#': number of image values (i.e. width x height x depth x spectrum). 
         . '+': sum of all pixel values. 
         . '-': difference of all pixel values. 
         . '*': product of all pixel values. 
         . '/': quotient of all pixel values. 
         . 'm': minimum pixel value. 
         . 'M': maximum pixel value. 
         . 'a': average pixel value. 
         . 'v': variance of pixel values. 
         . 't': text string built from the image values, regarded as ascii codes. 
         . 'c': (x,y,z,c) coordinates of the minimum value, separated by commas ','. 
         . 'C': (x,y,z,c) coordinates of the maximum value, separated by commas ','. 
         . '(x[%],_y[%],_z[%],_c[%],_boundary)': pixel value at (x[%],y[%],z[%],c[%]), with 
            specified boundary conditions { 0=dirichlet | 1=neumann | 2=cyclic }. 
         . Any other 'feature' is considered either as a specified subset of image values, or 
            as a mathematical expression to evaluate (associated to selected image). 
            For instance, '@{-1,0-50%}' is substituted by the sequence of numerical values 
            coming from the first half data of the last image, separated by commas ','. 
            Expression '@{0,w+h}' is substituted by the sum of the width and height of the 
            first image. 
    . '@!' is substituted by the visibility state of the instant display window [0] 
       (can be { 0=closed | 1=visible }). 
    . '@{!,feature}' or '@{!indice,feature}' is substituted by a specific feature of the 
       instant display window [0] (or [indice], if specified). Requested 'feature' can be: 
         . 'w': display width (i.e. width of the display area managed by the window). 
         . 'h': display height (i.e. height of the display area managed by the window). 
         . 'wh': display width x display height. 
         . 'd': window width (i.e. width of the window widget). 
         . 'e': window height (i.e. height of the window widget). 
         . 'de': window width x window height. 
         . 'u': screen width (actually independent on the window size). 
         .' v': screen height (actually independent on the window size). 
         . 'uv': screen width x screen height. 
         . 'x': X-coordinate of the mouse position (or -1, if outside the display area). 
         . 'y': Y-coordinate of the mouse position (or -1, if outside the display area). 
         . 'b': state of the mouse buttons { 1=left-but. | 2=right-but. | 4=middle-but. }. 
         . 'o': state of the mouse wheel. 
         . 'k': decimal code of the pressed key if any, 0 otherwise. 
         . 'n': current normalization type of the instant display. 
         . 'c': boolean (0 or 1) telling if the instant display has been closed recently. 
         . 'r': boolean telling if the instant display has been resized recently. 
         . 'm': boolean telling if the instant display has been moved recently. 
         . Any other 'feature' stands for a keycode name in capital letters, and is substi- 
            tuted by a boolean describing the current key state { 0=pressed | 1=released }. 
    . '@{"command line"}' is substituted by the status value set by the execution of the 
       specified command line (see command '-status'). 
    . Expression '@{}' stands thus for the current status value. 
  - '$name' and '${name}' are both substituted by the value of the specified named variable 
     (set previously by item 'name=value'), or by the current positive indice of the named 
     image '[name]', or by the value of the named OS environment variable (in this order). 
  - '$>' and '$<' (resp. '${>}' and '${<}') are shortcuts respectively for '@{>,-1}' and 
     '@{<,-1}'. They refer to the increasing/decreasing indice of the latest (currently 
     running) 'repeat..done' loop. 
  - Any other expression inside braces (as in '{expression}') is considered as a mathematical 
     expression, and is evaluated, except for the three following cases: 
    . If expression starts and ends by single quotes, it is substituted by the sequence of 
       ascii codes that composes the specified string, separated by commas ','. For instance, 
       item '{'foo'}' is substituted by '102,111,111'. 
    . If expression starts and ends with backquotes '`', it is substituted by the string 
       whose ascii codes are given by the list of values in between the backquotes. 
       For instance, item '{`102,111,111`}' is substituted by 'foo'. 
    . If expression contains operator ''=='' or ''!='', it is substituted by 0 or 1, whether 
       the strings beside the operator are the same or not (case-sensitive). For instance, 
       both items '{foo'=='foo}' and '{foo'!='FOO}' are substituted by '1'. 
    . If expression starts with an underscore '_', it is substituted by the mathematical 
       evaluation of the expression, truncated to a readable format. 
  - Item substitution is never done in items between double quotes. One must break the quotes 
    to enable substitution if needed, as in "3+8 kg = "{3+8}" kg". Using double quotes 
    is then a convenient way to disable the substitutions mechanism in items, when necessary. 
  - One can also disable the substitution mechanism on items outside double quotes, by 
     escaping the '@','{','}' or '$' characters, as in '\{3+4\}\ doesn't\ evaluate'.
\end{lstlisting}
\normalsize
\section{Mathematical expressions}
\small
\begin{lstlisting}
  - G'MIC has an embedded mathematical parser. It is used to evaluate expressions inside 
     braces '{}', or formulas in commands that may take one as an argument (e.g. '-fill'). 
  - When used in commands, a formula is evaluated for each pixel of the selected images. 
  - The mathematical parser understands the following set of functions, operators and variables: 
    _ Usual operators: || (logical or), && (logical and), | (bitwise or), & (bitwise and), 
       !=, ==, <=, >=, <, >, << (left bitwise shift), >> (right bitwise shift), -, +, *, /, 
       % (modulo), ^ (power), ! (logical not), ~ (bitwise not). 
    _ Usual functions: sin(), cos(), tan(), asin(), acos(), atan(), sinh(), cosh(), tanh(), 
       log(), log2(), log10(), exp(), sign(), abs(), atan2(), round(), narg(), arg(), 
       isval(), isnan(), isinf(), isint(), isbool(), rol() (left bit rotation), 
       ror() (right bit rotation), min(), max(), med(), kth(), sinc(), int(). 
       Function 'atan2()' is the version of atan() with two arguments 'y,x' (as in C/C++). 
       Function 'narg()' returns the number of specified arguments. 
       Function 'arg(i,a_1,..,a_n)' returns the ith argument a_i. 
       Functions 'min()', 'max()', 'med()' and 'kth()' can be called with an arbitrary number of arguments. 
       Functions 'isval()', 'isnan()', 'isinf()', 'isbool()' can be used to test the type of 
       a given number or expression. 
    _ The variable names below are pre-defined. They can be overloaded if necessary. 
         . 'w': width of the associated image, if any (0 otherwise). 
         . 'h': height of the associated image, if any (0 otherwise). 
         . 'd': depth of the associated image, if any (0 otherwise). 
         . 's': spectrum of the associated image, if any (0 otherwise). 
         . 'x': current processed column of the associated image, if any (0 otherwise). 
         . 'y': current processed row of the associated image, if any (0 otherwise). 
         . 'z': current processed slice of the associated image, if any (0 otherwise). 
         . 'c': current processed channel of the associated image, if any (0 otherwise). 
         . 'e': value of e, i.e. 2.71828.. 
         . 'pi': value of pi, i.e. 3.1415926.. 
         . '?' or 'u': a random value between [0,1], following a uniform distribution. 
         . 'g': a random value, following a gaussian distribution of variance 1 
            (roughly in [-5,5]). 
         . 'i': current processed pixel value (i.e. value located at (x,y,z,c)) of the 
            associated image, if any (0 otherwise). 
         . 'im','iM','ia','iv': Respectively the minimum, maximum, average values and 
            variance of the associated image, if any (0 otherwise). 
         . 'xm','ym','zm','cm': The pixel coordinates of the minimum value in the associated 
            image, if any (0 otherwise). 
         . 'xM','yM','zM','cM': The pixel coordinates of the maximum value in the 
            associated image, if any (0 otherwise). 
    _ These special operators can be used: 
         . ';': expression separator. The returned value is always the last encountered 
            expression. For instance expression '1;2;pi' is evaluated as 'pi'. 
         . '=': variable assignment. Variables in mathematical parser can only refer to. 
            numerical values. Variable names are case-sensitive. Use this operator in 
            conjunction with ';' to define complex evaluable expressions, such as 
             't=cos(x);3*t^2+2*t+1'. 
            These variables remain local to the mathematical parser and cannot be accessed 
            outside the evaluated expression. 
    _ The following specific functions are also defined: 
         . 'if(expr_cond,expr_then,expr_else)': return value of 'expr_then' or 'expr_else', 
            depending on the value of 'expr_cond' (0=false, other=true). For instance, 
            G'MIC command '-fill if(x%10==0,255,i)' will draw blank vertical lines on every 
            10th column of an image. 
         . '?(max)' or '?(min,max)': return a random value between [0,max] or [min,max], 
            following a uniform distribution. 'u(max)' and 'u(0,max)' mean the same. 
         . 'i(_a,_b,_c,_d,_interpolation,_boundary)': return the value of the pixel located 
            at position (a,b,c,d) in the associated image, if any (0 otherwise). 
            Interpolation parameter can be { 0=nearest neighbor | other=linear }. 
            Boundary conditions can be { 0=dirichlet | 1=neumann | 2=cyclic }. 
            Omitted coordinates are replaced by their default values which are respectively 
            x, y, z, c and 0. 
         . 'j(_dx,_dy,_dz,_dc,_interpolation,_boundary)': does the same for the pixel located 
            at position (x+dx,y+dy,z+dz,c+dc). 
         . 'i[offset]': return the value of the pixel located at specified offset in the associated 
            image buffer. 
         . 'j[offset]': does the same for an offset relative to the current pixel (x,y,z,c). 
            For instance command '-fill 0.5*(i(x+1)-i(x-1))' will estimate the X-derivative 
            of an image with a classical finite difference scheme. 
         . If specified formula starts with '>' or '<', the operators 'i(..)' and 'j(..)' will return 
            values of the image currently being modified, in forward ('>') or backward ('<') order. 
  - The last image of the list is always associated to the evaluations of '{expressions}', 
     e.g. G'MIC sequence '256,128 -f {w}' will create a 256x128 image filled with value 256.
\end{lstlisting}
\normalsize
\section{Image and data viewers}
\small
\begin{lstlisting}
  - G'MIC has some very handy embedded visualization modules, for 1d signals 
     (command '-plot'), 1d/2d/3d images (command '-display') and 3d objects 
     (command '-display3d'). It enables an interactive view of the selected image data. 
  - The following keyboard shortcuts are available in the interactive viewers: 
    . CTRL+D: Increase window size. 
    . CTRL+C: Decrease window size. 
    . CTRL+R: Reset window size. 
    . CTRL+F: Toggle fullscreen mode. 
    . CTRL+S: Save current window snapshot as numbered file 'gmic_xxxx.bmp'. 
    . CTRL+O: Save current instance of the viewed data, as numbered file 'gmic_xxxx.cimgz'. 
  - Shortcuts specific to the 1d/2d/3d image viewer are: 
    . CTRL+A: Switch cursor mode. 
    . CTRL+P: Play z-stack of frames as a movie (for volumetric 3d images). 
    . CTRL+V: Show/hide 3D view (for volumetric 3d images). 
    . CTRL+(mousewheel): Zoom in/out. 
    . SHIFT+(mousewheel): Go left/right. 
    . ALT+(mousewheel): Go up/down. 
    . Numeric PAD: Zoom in/out (+/-) and move through zoomed image (digits). 
    . BACKSPACE: Reset zoom scale. 
  - Shortcuts specific to the 3d object viewer are: 
    . (mouse)+(left mouse button): Rotate 3d object. 
    . (mouse)+(right mouse button): Zoom 3d object. 
    . (mouse)+(middle mouse button): Shift 3d object. 
    . (mousewheel): Zoom in/out. 
    . CTRL+F1 .. CTRL+F6: Switch between different 3d rendering modes. 
    . CTRL+Z: Enable/disable z-buffered rendering. 
    . CTRL+A: Show/hide 3d axes. 
    . CTRL+G: Save 3d object, as numbered file 'gmic_xxxx.off'. 
    . CTRL+T: Switch between single/double-sided 3d modes.
\end{lstlisting}
\normalsize
\section{Adding custom commands}
\small
\begin{lstlisting}
  - Custom commands can be defined by a user, through the use of G'MIC custom commands files. 
  - A command file is a simple ascii text file, where each line starts either by 
     'command_name: command_definition' or 'command_definition (continuation)'. 
  - Custom command names must use characters [a-zA-Z0-9_] and cannot start with a number. 
  - Any ' # comment' expression found in a custom commands file is discarded by the G'MIC 
     interpreter, wherever it is located in a line. 
  - In custom commands, the following $-expressions are substituted: 
    . '$*' is substituted by a copy of the specified string of arguments. 
    . '$"*"' is substituted by a copy of the specified string of arguments, each being around double quotes. 
    . '$#' is substituted by the maximum indice of known arguments (either specified by the 
       user or set to a default value in the custom command). 
    . '$?' is substituted by a string telling about the command subset restriction (only 
       useful when custom commands need to output descriptive messages). 
    . '$i' and '${i}' are both substituted by the i-th specified argument. Negative indices 
       such as '${-j}' are allowed and refer to the j^th latest argument. '$0' is substituted 
       by the custom command name. 
    . '${i=default}' is substituted by the value of $i (if defined) or by its new value set 
        to 'default' otherwise ('default' may be a $-expression as well). 
    . '${subset}' is substituted by the arguments values (separated by commas ',') of a 
       specified argument subset. For instance expression '${2--2}' is substitued by all 
       specified arguments except the first and the last one. Expression '${^0}' is then 
       substituted by all arguments of the invoked command (eq. to '$*' if all specified 
       arguments have indeed a value). 
    . '$=var' is substituted by the set of instructions that will assign each argument $i 
       to the named variable 'var$i' (for i in [0..$#]). This is particularly useful when a 
       custom command want to manage variable numbers of arguments. Variables names must 
       use characters [a-zA-Z0-9_] and cannot start with a number. 
  - These particular $-expressions are always substituted, even in double quoted items or 
     when the dollar sign '$' is escaped with a backslash '\'. To avoid substitution, place 
     an empty double quoted string just after the '$' (as in '$""1'). 
  - Specifying arguments may be skipped when invoking a custom command, by replacing them by 
     commas ',' as in expression '-flower ,,3'. Omitted arguments are set to their default 
     values, which must be thus explicitly defined in the code of the corresponding custom 
     command (using default argument expressions as '${1=default}'). 
  - If one numbered argument requested in a custom command has no value, an error is thrown 
     by the interpreter.
\end{lstlisting}
\normalsize
\section{List of commands}
\small
\begin{lstlisting}
 All available G'MIC commands are listed below, classified by themes. 
 When several choices of command arguments are possible, they appear separated by '|'. 
 An argument specified inside '[]' or starting by '_' is optional except when standing for an 
 existing image [image], where 'image' can be either an indice number or an image name. 
 In this case, the '[]' characters are mandatory when writing the item. A command marked with 
 '(*)' or '(+)' is a native command. '(*)' means the command is available for all pixel types, 
 otherwise only for the default 'float' pixel type. 
 Remember that native commands run faster than custom commands, so use then when possible. 
 Note also that all images in this reference documentation are normalized in [0,255] before 
 being displayed. You may need to do this manually (command '-normalize 0,255') if you want 
 save image files having the same aspect than those displayed.
\end{lstlisting}
\normalsize

\chapter{List of commands}

\section{Global options}


\subsection{\emph{-debug\index{-debug}} (*)}\vspace*{-0.5em}
Activate debug mode.
~\\When activated, the G'MIC interpreter becomes very verbose and outputs additionnal log
messages about its internal state on the standard output (stdout).
~\\This option can be useful when debugging the execution of a custom command.


\subsection{\emph{-help\index{-help}} (*)}\vspace*{-0.5em}
~\\\textbf{Arguments: } 
{\small \texttt{\_command}}~~~$|$\\
\hspace*{2.2cm}{\small \texttt{(no args)}}\\~\\
Display help (optionally for specified command only) and exit.
~\\(\emph{eq. to} {\small \texttt{'-h'}}).


\subsection{\emph{-version\index{-version}} }\vspace*{-0.5em}
Display current version number and exit.

\section{Inputs/outputs}


\subsection{\emph{-apply\_camera\index{-apply\_camera}} }\vspace*{-0.5em}
~\\\textbf{Arguments: } 
{\small \texttt{\_command,\_camera\_index$>$=0,\_skip\_frames$>$=0,\_output\_filename}}\\~\\
Apply specified command on live camera stream, and display it on display window [0].
~\\~\\\textbf{Default values}: {\small \texttt{'command=""', 'camera\_index=0' (default camera), 'skip\_frames=0'} and \texttt{'filename=""'.}}


\subsection{\emph{-apply\_files\index{-apply\_files}} }\vspace*{-0.5em}
~\\\textbf{Arguments: } 
{\small \texttt{"command",list\_of\_filenames,\_output\_prefix,\_output\_extension,\_view\_window=\{ 0 ~$|$~ 1 \}}}\\~\\
Apply specified command on all specified image files, by reading them one by one,
and save result by appending 'output\_prefix' to each original filename.
~\\'list\_of\_filenames' must be the list of filenames, separated by space.
~\\Thus, a specified filename cannot contain a spaces.
~\\If 'output\_extension' is set, the output files are written using the specified extension instead of keeping
the original one.
~\\~\\\textbf{Default value}: {\small \texttt{'output\_prefix=gmic\_', 'output\_extension=""'} and \texttt{'view\_window=0'.}}


\subsection{\emph{-camera\index{-camera}} (*)}\vspace*{-0.5em}
~\\\textbf{Arguments: } 
{\small \texttt{\_camera\_index$>$=0,\_nb\_frames$>$0,\_skip\_frames$>$=0,release\_camera=\{ 0 ~$|$~ 1 \},\_capture\_width$>$=0,\_capture\_height$>$=0}}\\~\\
Insert one or several frames from specified camera, with custom delay between frames (in ms).
~\\When 'release\_camera==1', the camera stream is released instead of capturing new images.
~\\~\\\textbf{Default values}: {\small \texttt{'camera\_index=0' (default camera), 'nb\_frames=1', 'skip\_frames=0', 'release\_camera=0'} and \texttt{'capture\_width=capture\_height=0' (default size).}}


\subsection{\emph{-command\index{-command}} (*)}\vspace*{-0.5em}
~\\\textbf{Arguments: } 
{\small \texttt{\_add\_debug\_info=\{ 0 ~$|$~ 1 \},\{ filename ~$|$~ http[s]://URL ~$|$~ "string" \}}}\\~\\
Import G'MIC custom commands from specified file, URL or string.
~\\(\emph{eq. to} {\small \texttt{'-m'}}).
~\\Imported commands are available directly after the '-command' invocation.
~\\~\\\textbf{Default value}: {\small \texttt{'add\_debug\_info=1'.}}
[gmic]-0./__help/document_gmic/../_document_gmic_example_latex/*if/*local/foo/ Input file 'image.jpg' at position [0]
[gmic]-0./__help/document_gmic/../_document_gmic_example_latex/*if/*local/foo/ *** Error *** Unknown filename 'image.jpg'.
[gmic]-1./__help/document_gmic/*repeat/ Abort G'MIC instance.

[gmic] *** Error in ./__help/document_gmic/../_document_gmic_example_latex/*if/*local/foo/ (file '../src/gmic_def.gmic', call from line 1882) *** Unknown filename 'image.jpg'.

